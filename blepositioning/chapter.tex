\chapter{BLE as positioning tool}

Before talking about certain details of using BLE for positioning, it is worth explaining why I think that BLE is a good candidate for indoor positioning.
Bluetooth defines profiles as part of its specification; these are standard protocols for standard use cases.
There is no requirement that a BLE device implements one or more of these profiles, however using a standard profile for a certain functionality increases interconnectability between devices.
One of the standard BLE profiles is the \define{proximity profile (PXP)}\urlref{https://developer.bluetooth.org/TechnologyOverview/Pages/PXP.aspx}.
The proximity profile allows an alert to be fired when a device loses connection, or if the \define{path loss}, the difference between transmitted power level and received power level, passes a threshold.
In this case the path loss is used as a measure of distance.

In 2013 with iOS 7 Apple introduced beacon regions.
Using an iBeacon (a BLE beacon with a certain payload), iOS \apps can be alerted when they get in or out of range, and which distance they have to the beacon.
This distance is based on the \define{received signal strength (RSS)}, which is closely related to path loss.
iBeacons can not be used (at the moment) for absolute positioning, since no information is known about the absolute position of the iBeacons.

\section{Techniques}
\define{Radiodetermination}, the technique of determining a device's location, can be done in two ways.
Either the environment can send radio signals, allowing the device to determine its own location (\define{passive positioning}), or the device can send signals that allow the environment to locate the device, possibly sending the calculated position back to the device (\define{active positioning}).
Note that this distinction is made on how the location of the device relative to the environment is made.
A system such as \wifi positioning is considered passive positioning, even if, after receiving the \wifi signals, a lookup on a server is needed to determine the actual location.

Several properties of radio signals can be used to do radiodetermination.
In the previous section I discussed path loss and RSS\footnote{
    received signal strength = sending signal strength - path loss.
    Since the sending signal strength is constant, path loss and RSS differ by a constant.
    Even though RSS itself does not tell us much, the difference between two RSS measurements gives the same information as the difference between two path loss measurements.
    In this report I will mostly use RSS, since this can be read directly by the receiver.
    If needed the transmitter can be configured to broadcast its sending signal strength, to calculate path loss; in these cases this will be mentioned separately.
}.
Alternatively one can use time-of-flight, the fact that a longer path between transmitter and receiver means that a radio signal takes longer to reach the receiver.
This technique is being used in GNSS.
One can also determine the angle from which a signal was received, using radio-direction-finding.

\subsection{Passive positioning}
In passive positioning a device positions itself based on radio signals from the environment.
In our case this means an environment with multiple BLE beacons that transmit at a regular interval



There are other methods how radio waves can be used to determine a position, for instance using time-of-flight (GNSS use this method), or radio-direction-finding.
These methods could possibly be used for BLE location 


If a beacon's sending power is known, one can easily calculate the path loss if the RSS is measured; if the sending power is not known, it can be assumed to be constant, and the path loss and RSS differ by a constant value.
Since 

 and, indirectly, on RSS (path loss is send power - RSS, so assuming send-power is constant, these two give the same information).

If we assume that a positioning system based on BLE is technologically possible (all architectures proposed here assume that we can infer some data about our position by examining the path loss between the 
