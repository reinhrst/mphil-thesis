Where only a couple of decades ago, positioning one's self meant using a paper map, or asking people for directions, two innovations have resulted in most people now carrying a positioning device with them wherever they go.
First GPS  blalala


GPS and other GSNS work well in areas where a large area of the sky can be seen, however they work less well in cities with lots of high rise buildings, or indoors.
In addition, because of its global nature, GSNS are limited to a global coordinate space; there are situations, such as when on a ship at sea, that one's position relative to the ship is more important than the exact position on the globe.
Finally the error in GSNS positioning is in many cases good enough for outdoor positioning, where streets are tens of meters away from one another, however indoors an error of several meters may be the difference between the reception, the restaurant or the neighbouring building.

Much research has been done into indoor positioning. There is active Bat blablabablabla.
An area of particular interest here is \wifi positioning (radar, blala).
\Wifi positioning uses the existing infrastructure of \wifi access points, and relies for reception on standard \wifi client hardware, that is already available in most handheld devices.
As a result, WPS (\Wifi Positioning System) is being used in most smartphones in order to augment other positioning systems.

In order to do \wifi positioning, a survey is made of an area, measuring \wifi access points' signal strength at different locations.
When a smartphone tries to determine its position, the received signal strength (RSS) is compared to the signal strength at the surveyed locations, and a match is made.
Several methods exist on how to exactly make the match, and \ref{EKNN} shows that different results can be obtained using these methods.



In July 2013 Apple introduced the iBeacon technology.
An iBeacon sends Bluetooth Low Energy (BLE) packets at regular intervals, and a BLE capable device (such as an iPhone) can receive them and determine its proximity to the beacon.
This allows for accurate absolute positioning (within a meter), but the technology is at the moment only being used to determine whether a device is close to a certain beacon; since the beacon's position itself is not known however, this technique does not improve absolute positioning.






